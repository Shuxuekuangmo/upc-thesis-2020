\documentclass[../../main.tex]{subfiles}
\begin{document}
% 一个文件最好只写一个章节
\section{章节名称}
% 章节内容



\section{线性表的基本理论知识}
\subsection{线性表的定义}
线性表是最简单、最常用的一种数据结构。线性表\cite{1}是$n (n\ge 0)$个数据元素的有限序列。

……。
\subsection{线性顺序表}
线性表的顺序存储结构的特点是为表中相邻的元素$a_i$和$a_{i+1}$ 赋以相邻的存储位置。
\subsubsection{三级标题名}
\subsubsection{三级标题名}
\begin{itemize}
        \item [(1)] 三级以下标题
\end{itemize}




\subsection{线性链表}

线性表的链式存储结构的特点是用一组任意的存储单元存储线性表的数据元素(这组元素可以是连续的,也可以是不连续的)。


\begin{definition}
线性表的定义为
\end{definition}


\begin{theorem}
设sxibaxabxugbuaxbsusabx
\begin{proof}
由线性表的特殊定义可得。
\end{proof}
\end{theorem}
设sxibaxabxugbuaxbsusabx
\begin{lemma}
设sxibaxabxugbuaxbsusabx
\begin{proof}
由线性表的特殊定义可得。
\end{proof}
\end{lemma}






\end{document}
