\documentclass[../../main.tex]{subfiles}
\begin{document}
% 一个文件最好只写一个章节
\section{章节名称}
% 章节内容



\section{定理、证明类环境}\label{sec:1}
正文字体\par
中文字体测试 english font test
\begin{definition}
中文字体测试 english font test\par
断行测试断行测试断行测试断行测试断行测试断行测试断行测试断行测试断行测试断行测试断行测试断行测试断行测试断行测试断行测试断行测试\par
\end{definition}

\begin{theorem}
中文字体测试 english font test\par
断行测试断行测试断行测试断行测试断行测试断行测试断行测试断行测试断行测试断行测试断行测试断行测试断行测试断行测试断行测试断行测试\par
\end{theorem}

%带定理名的定理
\begin{theorem}[勾股定理]
中文字体测试 english font test\par
断行测试断行测试断行测试断行测试断行测试断行测试断行测试断行测试断行测试断行测试断行测试断行测试断行测试断行测试断行测试断行测试\par

\begin{proof}
中文字体测试 english font test\par
行间公式$E=mc^2$测试\par
断行测试断行测试断行测试断行测试断行测试断行测试断行测试断行测试断行测试断行测试断行测试断行测试断行测试断行测试断行测试断行测试\par
\begin{equation}
	E=mc^2
\end{equation}
\end{proof}


\begin{solution}
中文字体测试 english font test\par
行间公式$E=mc^2$测试\par
断行测试断行测试断行测试断行测试断行测试断行测试断行测试断行测试断行测试断行测试断行测试断行测试断行测试断行测试断行测试断行测试\par
\begin{equation}
\label{eq:1}
	E=mc^2
\end{equation}
\end{solution}


\autoref{eq:1}是质能转换方程,\autoref{sec:1}是测试文档各种环境的章节
\end{theorem}




\begin{lemma}
中文字体测试 english font test\par
断行测试断行测试断行测试断行测试断行测试断行测试断行测试断行测试断行测试断行测试断行测试断行测试断行测试断行测试断行测试断行测试\par
\end{lemma}


\begin{corollary}
中文字体测试 english font test\par
断行测试断行测试断行测试断行测试断行测试断行测试断行测试断行测试断行测试断行测试断行测试断行测试断行测试断行测试断行测试断行测试\par
\end{corollary}

\begin{example}
中文字体测试 english font test\par
断行测试断行测试断行测试断行测试断行测试断行测试断行测试断行测试断行测试断行测试断行测试断行测试断行测试断行测试断行测试断行测试\par
\end{example}

\begin{remark}
中文字体测试 english font test\par
断行测试断行测试断行测试断行测试断行测试断行测试断行测试断行测试断行测试断行测试断行测试断行测试断行测试断行测试断行测试断行测试\par
\end{remark}







\end{document}