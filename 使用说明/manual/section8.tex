\documentclass[../../main.tex]{subfiles}
\begin{document}
% 一个文件最好只写一个章节
\section{代码环境}
% 章节内容

伪代码、算法环境,内部可以使用数学模式来表达。
\begin{algorithm}[!h]
	\caption{PARTITION$(A,p,r)$}%算法标题
	
	\label{11111111}

	\begin{algorithmic}[1]%一行一个标行号


		\STATE $i=p$
		\FOR{$j=p$ to $r$}
		\IF{$A[j]<=0$}
		\STATE $swap(A[i],A[j])$//交换两个值
		\STATE $i=\sum_{j=0}^{n} x^k$
		\ENDIF
		\ENDFOR
	\end{algorithmic}
\end{algorithm}


\autoref{11111111}

代码环境,适合简短代码的情况。

\begin{lstlisting}{c}
int main(int argc, char const *argv[])
{
	/* code */
	return 0;
}
\end{lstlisting}

从文件中插入代码(推荐),相当于对代码环境做了封装,直接从文件中导入代码。建议将代码文件放在主目录的code/目录内,然后通过\verb|\lstinputlisting{路径}|命令将代码引入。注意使用相对路径。

这样的好处是代码修改后可以直接编译,省去了复制粘贴的步骤,能有效防止因疏忽导致的最终论文中代码与实际代码不一致的情况。
\lstinputlisting{../../code/test.cpp}

\end{document}




