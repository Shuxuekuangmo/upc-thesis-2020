\documentclass[../../main.tex]{subfiles}
\begin{document}


\section{基础知识}
\subsection{运行latex}
安装好texlive环境后,使用xelatex引擎编译main.tex文件来得到
\subsection{组织章节、目录}
一些简单的命令:



\subsection{简单的文本组织}

以符号\verb|%|开头的行是注释。LaTeX会完全忽略百分号后的内容,不使其输出到产生的
pdf文件中。


latex 会自动对段落进行缩进,并且会忽略掉段前的空格。中文句子中的空格会
被全部忽略,英文句子中两个单词之间的多个空格会被视为一个空格,多个空
格也不会加大单词之间的间距。中英文进行混排时,xelatex 也会自动的处理空格。

示例:

源文件中的原文:

\verb|中      国    石油     大学China     University      Of    Petroleum|

实际的打印效果:\\
中      国    石油     大学China               University      Of    Petroleum



在latex中,单个换行(回车)其实就相当于一个空格,不会使文字另起一段,而是起使源代码更易阅读的作用。
使用两个反斜杠\verb|\\|或命令\verb|\newline|可以进行换行,但不会另起一段。
要另起一段,使用\verb|\par|命令或者使用两个及以上的换行(多敲几次回车)。

注意,多个换行只会起到分段作用,不会在文件中输出其他的空白行。这一点与word不同。

示例:

源文件中的原文:
\begin{verbatim}
一二三四五六七
一二三四五六七

一二三四五六七\\
一二三四五六七

一二三四五六七\par
一二三四五六七
\end{verbatim}

一二三四五六七
一二三四五六七

一二三四五六七\\
一二三四五六七

一二三四五六七\par
一二三四五六七

空白
hspace





\subsection{命令与环境}


LaTeX中的控制命令(宏)以反斜线开头,后跟命令名,并且可以传入一些参数。
用来实现一些强大的功能。



latex的许多样式依靠环境来实现


\subsubsection{公式与定理环境}

行内公式$E=mc^2$ 用两个美元符号将公式内容包裹起来即可


带编号的行间公式环境

不编号的行间公式环境
\[
e=mc^{2}
\]

\begin{equation}
E=mc2
\end{equation}
对编号的公式进行引用
\begin{equation}
  \label{eq:质能转换方程}
e=mc^{2}
\end{equation}
引用:\autoref{eq:质能转换方程}。





定理类环境

\begin{definition}\label{1}

定义环境
\end{definition}

\begin{theorem}\label{2}

定理环境
\end{theorem}

\autoref{thm:thm1}是一个定理

\begin{lemma}\label{3}
引理环境
\end{lemma}


\begin{proposition}\label{4}
命题环境
\end{proposition}



\begin{corollary}\label{5}
推论环境
\end{corollary}


\begin{example}\label{6}
例 环境
\end{example}






\begin{remark}\label{7}
\label{thm:thm1}
  注环境
\end{remark}




\autoref{1}
\autoref{2}
\autoref{3}
\autoref{4}
\autoref{5}
\autoref{6}
\autoref{7}

\begin{proof}
证明环境
\end{proof}


\begin{solution}
解环境
\end{solution}



\begin{theorem}

  证明可以放进定理类环境里面
\begin{proof}
这是证明环境
\end{proof}
\end{theorem}

\begin{example}
解也一样

\begin{solution}
这是解
\end{solution}

\end{example}


\subsection{多行公式}
在行间公式环境中包裹align 环境或 aligned环境,通过符号 来进行换行,符
号 来进行对齐

\begin{equation}
  \label{}
\begin{aligned}
  a&=b+c+d+e \\
  a+b&=c+d+e \\
\end{aligned}
\end{equation}




文本


\verb|\mathbf{}| 公式环境内的粗体
\verb|\textbf{}| 文本的粗体
\verb|\emph{}| 斜体强调
\verb|\hspace{1em}| 产生一段横向的空白,长度为当前字号下一个字符"m"的宽度。
\verb|\vspace{1em} |产生一段纵向的空白,长度为当前字号下一个字符"m"的高度。
一些常用的特殊字符被LaTeX作为控制字符,如果要实际的打印出他们,需要使用反斜杠\verb|\|进行转义。特殊字符包括\verb|\,#,%,$,&,_,^,{,},~|等。


\subsection{代码和抄录环境}



\subsection{图表环境}
比较复杂,另起一章说明
\end{document}