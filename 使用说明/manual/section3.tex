\documentclass[../../main.tex]{subfiles}
\begin{document}


\section{模板功能}
\subsection{打印模式}
论文答辩时需要准备三份纸质论文,而纸质论文都是双面打印。我们的论文都是单面排版的,打印的时候就需要在特定的位置加入空白页。

我们当时的要求是:在封面,中英摘要,目录,致谢,参考文献后添加空白页,并且空白页的位置必须是
\textbf{实际页码}的偶数页。纸质版论文除了空白页外其他部分与电子版论文完全一致。


本模板中暂时按此要求实现了双面打印宏\verb| \twoSidePrint{} |,定义在
在main.tex中导言区的位置,将其取消注释后编译即可获得双面打印效果支持,在需要的位置自动插入空白页。
如果后续要求变更,再按照实际的要求进行修改。

具体实现的命令和行为:

\verb|\ClearPageStyle|命令,若\verb|\twoSidePrint{} |未定义,无任何影响,否则添加空白页。用在无页码的封面页、中英摘要页、目录页之后。

\verb|\PrintModeSubfile|环境,若\verb|\twoSidePrint{} |未定义,仅导入一子文件,否则导入子文件后根据起始页码与结束页码判断此子文件的页数,若为奇数,添加一空白页,并且空白页不会编页,页码仍与单面打印时保持一致。直接作用在章节的导入过程。


注意事项:
一般来说,中英摘要都只有一页,目录可能会有多页。如果中英摘要或目录的页数是偶数,如2页或4页,\verb|\ClearPageStyle|命令插入的空白页就会成为多余的。这时你需要手动将它注释掉来去掉这一多余的空白页,达到你实际想要的打印效果。

\end{document}
