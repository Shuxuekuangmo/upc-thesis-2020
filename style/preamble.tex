%%% 产生一段随机文本的宏包
\usepackage{lipsum}
\usepackage{zhlipsum}


\usepackage{makecell}
%%% 数学环境?
\usepackage{amsmath}
\usepackage{mathtools}
\usepackage{amsfonts,amssymb}

%%% 子图
\usepackage{subfigure}

%%% 列表环境
\usepackage{enumerate}
%%% 图片环境
\usepackage{graphicx}

%%%暂时没发现有什么用,注释掉
% \usepackage{epstopdf}
\usepackage{etoolbox}
\usepackage{titlesec}


% \usepackage{tocloft}


%%% 令长表格自动换页的宏包
\usepackage{longtable}


\usepackage{subfiles}
%%% 定义定理环境的宏包
\usepackage{amsthm}
%%%审阅功能
% \usepackage{changes}
%%%交叉引用宏包,要放在最后引入
\usepackage{hyperref}




%%%%%---------------------------------------------------------------------
% 祖传代码,别乱动


%%%%%%%%%调整subsection与subsubsection格式
\titlespacing*{\subsubsection}{0pt}{0.5ex plus .2ex minus .2ex}{%
	0.5ex plus .2ex
}
\titlespacing*{\subsection}{0pt}{0.5ex plus .2ex minus .2ex}{%
	0.5ex plus .2ex
}
%%%%%%%%%貌似是调整参考文献格式?
\let\oldthebibliography \thebibliography
\let\endoldthebibliography \endthebibliography
%\renewenvironment{thebibliography}[1]{%
%	\begin{oldthebibliography}{#1}%
%		\setlength{\parskip}{0ex}%
%		\setlength{\itemsep}{0ex}%
%		\setlength{\itemindent}{4ex}
%		\setlength{\leftmargin}{-3pt}%
%	}% 
%	{
%	\end{oldthebibliography}%
%}
%%%%%%%%%调整数学公式格式,貌似是设置公式编号格式?
\renewcommand{\theequation}{\thesection -\arabic{equation}}
\makeatletter
%%%%%%%%%貌似是调整参考文献文本格式(如缩进等),非参考文献自身格式
\renewenvironment{thebibliography}[1]
{\section*{\refname}%
	\@mkboth{\MakeUppercase\refname}{\MakeUppercase\refname}%
	\list{\@biblabel{\@arabic\c@enumiv}}%
	{\settowidth\labelwidth{\@biblabel{#1}}%
		\setlength{\itemindent}{\dimexpr\labelwidth+\labelsep}
		\leftmargin\z@
		\@openbib@code
		\usecounter{enumiv}%
		\let\p@enumiv\@empty
		\renewcommand\theenumiv{\@arabic\c@enumiv}}%
	\sloppy
	\clubpenalty4000
	\@clubpenalty \clubpenalty
	\widowpenalty4000%
	\sfcode`\.\@m}
{\def\@noitemerr
	{\@latex@warning{Empty `thebibliography' environment}}%
	\endlist}



%%%%貌似是数学公式从每个章节(section)重新开始编号
\@addtoreset{equation}{section}








%%%%%---------------------------------------------------------------------
% 中英定理类环境的声明

% 定义定理样式
\newtheoremstyle{thesty}
{3pt} %环境前间距
{3pt} %环境后间距
{\songti} %定理内字体
{2em} %头部缩进
{\bfseries\songti} %定理头部字体
{} %头部后添加符号
{0.5em} %头部后间距
{} %theorem head spec

% 应用定理样式
\theoremstyle{thesty}

\ifx\enEnvironment\undefined
% 中文定理环境
{
	\newtheorem{definition}{定义}[section]
	\newtheorem{theorem}{定理}[section]
	\newtheorem{lemma}[theorem]{引理}
	\newtheorem{proposition}[theorem]{命题}
	\newtheorem{corollary}[theorem]{推论}
	\newtheorem{example}{例}[section]
	\newtheorem{remark}{注}[section]
}
\newenvironment{proofenv}[1]{\par\indent\songti\textbf{#1}\hspace{0.3em}}{\hfill\qedsymbol\par}
\renewenvironment{proof}{\begin{proofenv}{证明:}}{\end{proofenv}}
\newenvironment{solution}{\begin{proofenv}{解:}}{\end{proofenv}}

\else

%% 英文定理环境
{
	\newtheorem{theorem}{Theorem}[section]
	\newtheorem{lemma}[theorem]{Lemma}
	\newtheorem{proposition}[theorem]{Proposition}
	\newtheorem{corollary}[theorem]{Corollary}
	\newtheorem{definition}{Definition}[section]
	\newtheorem{example}{Example}[section]
	\newtheorem{remark}{Remark}[section]
}

\newenvironment{proofenv}[1]{\par\indent\songti\textbf{#1}\hspace{0.3em}}{\hfill\qedsymbol\par}
\renewenvironment{proof}{\begin{proofenv}{Proof:}}{\end{proofenv}}
\newenvironment{solution}{\begin{proofenv}{Solution:}}{\end{proofenv}}

\fi






%%%%%---------------------------------------------------------------------

% 交叉引用样式


\def\sectionautorefname~#1\null{%
	第~#1~节\null
}
\def\subsectionautorefname~#1\null{%
	第~#1~小节\null
}
\def\subsubsectionautorefname~#1\null{%
	第~#1~小节\null
}
\def\subsubsectionautorefname~#1\null{%
	第~#1~小节\null
}
\def\paragraphautorefname~#1\null{%
	段落~#1~\null
}
\def\subparagraphautorefname~#1\null{%
	段落~#1~\null
}


% 重新设置图表 auto ref
\def\figureautorefname~#1\null{%
	图~#1~\null
}
\def\tableautorefname~#1\null{%
	表~#1~\null
}

% 重新设置公式autoref
\def\equationautorefname~#1\null{%
	式~(#1)~\null
}












%%%%%---------------------------------------------------------------------

% 单双面打印控制

\ifx \twoSidePrint \undefined
    \def \ClearPageStyle{\clearpage}
\else
    \def \ClearPageStyle{
    		\clearpage
	    	\thispagestyle{empty}
	    	\quad
	    	\clearpage
    }
\fi

\newcounter{startpage}
\newcounter{endpage}
\setcounter{startpage}{1}
\setcounter{endpage}{1}

\newenvironment{PrintMode}{
	\clearpage
	\setcounter{startpage}{\value{page}}


}{	
	\clearpage
	\setcounter{endpage}{\value{page}}
	% \thestartpage
	% \theendpage
	\addtocounter{endpage}{\value{startpage}}

	\ifodd \value{endpage}
		\ClearPageStyle
		\ifx \twoSidePrint \undefined
		\else
			\addtocounter{page}{-1} % 如果环境内补了一页空白页,将实际页码减一
		\fi
	\fi


}

\newcommand \PrintModeSubfile[1]{\begin{PrintMode}\subfile{#1}\end{PrintMode}}



%%%%%---------------------------------------------------------------------


\ifx \eletronicDocument \undefined

\else
\hypersetup{
	% backref = page,
	% pagebackref = true,
	colorlinks = true,
	linkcolor = blue,
	citecolor = blue,
	urlcolor = blue,
	pdfborder = 000, %去掉链接红(黑)框
	% bookmarks = true,
	% bookmarksopen = true,
	bookmarksnumbered = true
}
\fi