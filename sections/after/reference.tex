\documentclass[../../main.tex]{subfiles}
\begin{document}


%\bibliography{./bibs/bibliography.bib}

%%%%%之前的.bib生成引用的样式与word要求不一致。能力有限,暂时无法修改
%%%%%样式文件不同于学校的要求。劳烦同学们手动引用文献

%%%%%手动指定在目录添加参考文献条目


% 按照学校word模板中对参考文献的要求,列出以下几点给同学们参考


% 列出的参考文献必须在正文中有引用,并且需按正文中出现的次序进行排序。同一文献出现多次,只用同一标号

% 参考文献里的标点符号均为英文格式输入,每个标点符号与后面的内容之间要空一格。参考文献的各项条目使用逗号分割,最后要有句点。


% 参考文献应不少于10篇(外文文献至少2篇,外语专业应以外文文献为主)。

% 文献引用的格式大致为:

% 作者1, 作者2, 作者3, et. al, 题目, 期刊, 时间, 期数(卷数), 起-始页,网址.

% 其中不那么重要的或没有的部分可不写
% 其中:

% 英文作者,名缩写(老外是名在前,姓在后),如:Robert Jort缩写为:R. Jort,名字两个单词的,G. H. Golub。作者太多不适合全部列出的,写上 et. al,

% 英文题目除专有名词外,仅第一个单词首字母大写
% 题目中表示文献类型的符号:[M] [J] 等一律删掉,不允许出现。
% 期刊名应写全称,不知道的可以上网搜索。英文期刊中实词首字母的写。
% 一些英文常见期刊:

% 日期统一改为如下格式 2003.5.12

\pagestyle{afterbody}
\phantomsection
\addcontentsline{toc}{section}{参考文献}

\begin{thebibliography}{99}

\bibitem{1} 严蔚敏, 吴伟民, 数据结构, 北京: 清华大学出版社, 1997.4.
\bibitem{2} 沈晴霓, 聂青, 苏京霞, 现代程序设计—C++与数据结构面向对象的方法与实现, 北京: 北京理工大学出版社, 2002.8.
\bibitem{3} T. Connolly, C. Begg, Database systems, 北京: 电子科技工业出版社, 2004.7.
\bibitem{4} R. Bate, S. Shrum, CMM Integration framework, CMU/SEI Spotlight, 1998, 4(3): 25-28.
\bibitem{5} J.P. Kuilboer, N. Ashrafi, Software process and product improvement, Physical Review A, 2000, 42(1): 27-34.
\bibitem{6} 张美金, 吴大伟, 基于ASP技术的远程教育系统体系结构的研究, http://172.50.0.88:86 /~cddbn/Y517807/pdf/index.htm, 2003-05-01.
\bibitem{7} 王伟国, 刘永萍, 王生年等, B/S模式网上考试系统分析与设计, 石河子大学学报(自然科学版), 2003, 6(2): 145-147.
\bibitem{8} …
\bibitem{9} …
\bibitem{10} …
\end{thebibliography}





\end{document}





