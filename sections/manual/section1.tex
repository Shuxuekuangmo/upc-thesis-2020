\section{数学环境}



行内公式$E=mc^2$


带编号的行间公式

\begin{equation}
E=mc2
\end{equation}

不编号的行间公式
\[
e=mc^{2}
\]

定理类环境

\begin{definition}
这一章是数学环境的使用样例。
\end{definition}

\begin{theorem}
这一章是数学环境的使用样例。
\end{theorem}


\begin{lemma}
这一章是数学环境的使用样例。
\end{lemma}


\begin{proposition}
这一章是数学环境的使用样例。

\end{proposition}



\begin{corollary}
这一章是数学环境的使用样例。
\end{corollary}


\begin{example}
这一章是数学环境的使用样例。
\end{example}




\begin{remark}
这一章是数学环境的使用样例。
\end{remark}





\begin{proof}
这一章是数学环境的使用样例。
\end{proof}


\begin{solution}
这一章是数学环境的使用样例。
\end{solution}



\begin{theorem}

  证明可以放进定理类环境里面
\begin{proof}
这是证明
\end{proof}
\end{theorem}

\begin{example}
解也一样

\begin{solution}
这是解
\end{solution}

\end{example}


\[
\left \{
\begin{aligned}
&\varepsilon \frac{\partial E}{\partial t}=\frac{\partial H}{\partial x},
E=E\left ( x,t \right ),H=H\left ( x,t \right ),\left ( x,t \right )\in \left [ 0,1 \right ]\times \left[ 0,1 \right ]\\
&\mu \frac{\partial H}{\partial t}=\frac{\partial E}{\partial x} \\
&E\left ( x,0 \right )=sin\left ( 2\pi x \right )\\
&H\left ( x,0 \right )=0\\
&E\left ( 0,t \right )=E\left ( 1,t \right ),H\left ( 0,t \right )=H\left ( 1,t \right )
\end{aligned}
\right.
\]\[
\left \{
\begin{aligned}
&\varepsilon \frac{\partial E}{\partial t}=\frac{\partial H}{\partial x},
E=E\left ( x,t \right ),H=H\left ( x,t \right ),\left ( x,t \right )\in \left [ 0,1 \right ]\times \left[ 0,1 \right ]\\
&\mu \frac{\partial H}{\partial t}=\frac{\partial E}{\partial x} \\
&E\left ( x,0 \right )=sin\left ( 2\pi x \right )\\
&H\left ( x,0 \right )=0\\
&E\left ( 0,t \right )=E\left ( 1,t \right ),H\left ( 0,t \right )=H\left ( 1,t \right )
\end{aligned}
\right.
\]