\documentclass[../../main.tex]{subfiles}
\begin{document}




\section{浮动体}

\subsection{表格环境}

普通的三线表

\begin{table}[htbp]
        \newcommand{\tabincell}[2]{\begin{tabular}{@{}#1@{}}#2\end{tabular}}
        \centering
        \caption{激光入射功率密度对导轨滚道表面硬化层深和显微硬度的影响}
        \begin{tabular}{ccccc}
                \toprule    %顶行线
                试验编号 & 功率密度 & 辐照时间 & 显微硬度       & 硬化层深\\ 
                \midrule    %中线
                t-1	&6.37×103	&0.067	&570,456	&0.354\\
                t-2	&6.37×103	&0.067	&570,456	&0.354\\
                t-3	&6.37×103	&0.067	&570,456	&0.354\\
                t-4	&6.37×103	&0.067	&570,456	&0.354\\
                t-5	&6.37×103	&0.067	&570,456	&0.354\\ 
          \bottomrule   %底行线
        \end{tabular}
        \label{data_table}
\end{table}


\autoref{data_table}

表格环境中有几个参数,h、t、b、p,这代表这个表格将会优先按照在页面的此
处、顶部、底部、 的顺序依次尝试插入这个表格,因为表格过大的时候可
能会造成在页面在已经有了文字的情况下剩余的空间不足以放下这个表格。


tabular环境的参数有c/l/r三种,分别代表此列元素按照中心/左/右对齐的格式
排版

small 命令使表格内部使用小五号字体
caption命令表示图标的标题,内部可以使用行间数学环境



当表格比较大时,\verb|\table|
环境就无法满足我们的要求了,








\end{document}