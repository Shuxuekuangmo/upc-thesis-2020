
\section{实验及结果分析}
例如由于起初未能真正掌握各种控件的功能,我设想是要一个下拉菜单,但是学识肤浅的我试了很多种就是达不到我要的效果,……。

……

关于……的影响如表\ref{data_table}所示。

……

\begin{table}[htbp]
        \small
        \newcommand{\tabincell}[2]{\begin{tabular}{@{}#1@{}}#2\end{tabular}}
        \centering
        \caption{激光入射功率密度对导轨滚道表面硬化层深和显微硬度的影响}
        \begin{tabular}{ccccc}
                \toprule
                试验编号 & 功率密度 & 辐照时间 & 显微硬度       & 硬化层深\\ \midrule
                t-1	&6.37×103	&0.067	&570,456	&0.354\\
                t-2	&6.37×103	&0.067	&570,456	&0.354\\
                t-3	&6.37×103	&0.067	&570,456	&0.354\\
                t-4	&6.37×103	&0.067	&570,456	&0.354\\
                t-5	&6.37×103	&0.067	&570,456	&0.354\\ \bottomrule
        \end{tabular}
        \label{data_table}
\end{table}


鉴于表格复杂性,此处提供了可换行示例表见表\ref{kehuanhang}
\begin{table}[htbp]
        \small
        \newcommand{\tabincell}[2]{\begin{tabular}{@{}#1@{}}#2\end{tabular}}
        \centering
        \caption{可换行示例表}
        \begin{tabular}{ccc}
                \toprule
                1	& 2& 3\\ \midrule
                1&\tabincell{c}{3}&6\\
                1&\tabincell{c}{3}&6\\
                \tabincell{c}{2}&\tabincell{c}{4444444444\\5555555555}&\tabincell{c}{6}
                \\ \bottomrule
        \end{tabular}
        \label{kehuanhang}
        \vspace{0.5em}
\end{table}

此处也提供了多列合并示例表如表\ref{duoliehebing}
\begin{table}[htbp]
        \small
        \centering
        \caption{多列合并示例表}
        \begin{tabular}{ccccccccc}
                \toprule
                & \multicolumn{2}{c}{ZZ}& \multicolumn{6}{c}{XX}\\
                \cmidrule(lr){2-3} \cmidrule(lr){4-9}
                &   &   & \multicolumn{2}{c}{CC}&\multicolumn{4}{c}{VV}\\
                \cmidrule(lr){4-5} \cmidrule(lr){6-9}
                &   &   &   &   & \multicolumn{3}{c}{BB}&NN\\
                \cmidrule(lr){6-8} \cmidrule(lr){9-9}
                & A &S	&D &F &G &H &J &K \\ \midrule
                Q&$\surd$&$\surd$&   &   &   &    & &\\
                T&	&		& $\surd$ & $\surd$   &   &  &$\surd$     &\\
                Y&	&		& $\surd$ & $\surd$   &   &  &    &$\surd$ \\ \bottomrule
        \end{tabular}
        \label{duoliehebing}
\end{table}
