
\ifx\compileAllFiles\undefined
\documentclass[supercite]{upcthesis}
\usepackage{lipsum}
\usepackage{makecell}
\usepackage{amsmath}
\usepackage{mathtools}
\usepackage{amsfonts,amssymb}
\usepackage{subfigure}
\usepackage{enumerate}
\usepackage{graphicx}
\usepackage{epstopdf}
\usepackage{etoolbox}
\usepackage{titlesec}
\usepackage{tocloft}

\usepackage{amsthm}



% \def\entheorem{} %%%%%将此行取消注释以切换为英文定理环境



%%%%%中英定理类环境的声明
%%%%%---------------------------------------------------------------------

% 定义定理样式
\newtheoremstyle{thesty}
{3pt} %环境前间距
{3pt} %环境后间距
{\songti} %定理内字体
{2em} %头部缩进
{\bfseries\songti} %定理头部字体
{} %头部后添加符号
{0.5em} %头部后间距
{} %theorem head spec

% 应用定理样式
\theoremstyle{thesty}

\ifx\entheorem\undefined
% 中文定理环境
{
	\newtheorem{definition}{定义}[section]
	\newtheorem{theorem}{定理}[section]
	\newtheorem{lemma}[theorem]{引理}
	\newtheorem{proposition}[theorem]{命题}
	\newtheorem{corollary}[theorem]{推论}
	\newtheorem{example}{例}[section]
	\newtheorem{remark}{注}[section]
}
\newenvironment{proofenv}[1]{\par\indent\songti\textbf{#1}\hspace{0.3em}}{\hfill\qedsymbol\par}
\renewenvironment{proof}{\begin{proofenv}{证明:}}{\end{proofenv}}
\newenvironment{solution}{\begin{proofenv}{解:}}{\end{proofenv}}

\else

%% 英文定理环境
{
	\newtheorem{theorem}{Theorem}[section]
	\newtheorem{lemma}[theorem]{Lemma}
	\newtheorem{proposition}[theorem]{Proposition}
	\newtheorem{corollary}[theorem]{Corollary}
	\newtheorem{definition}{Definition}[section]
	\newtheorem{example}{Example}[section]
	\newtheorem{remark}{Remark}[section]
}

\newenvironment{proofenv}[1]{\par\indent\songti\textbf{#1}\hspace{0.3em}}{\hfill\qedsymbol\par}
\renewenvironment{proof}{\begin{proofenv}{Proof:}}{\end{proofenv}}
\newenvironment{solution}{\begin{proofenv}{Solution:}}{\end{proofenv}}

\fi
%%%%%---------------------------------------------------------------------




\def\sectionautorefname~#1\null{%
	第~#1~节\null
}
\def\subsectionautorefname~#1\null{%
	第~#1~小节\null
}
\def\subsubsectionautorefname~#1\null{%
	第~#1~小节\null
}
\def\subsubsectionautorefname~#1\null{%
	第~#1~小节\null
}
\def\paragraphautorefname~#1\null{%
	段落~#1~\null
}
\def\subparagraphautorefname~#1\null{%
	段落~#1~\null
}


% 重新设置图表 auto ref
\def\figureautorefname~#1\null{%
	图~#1~\null
}
\def\tableautorefname~#1\null{%
	表~#1~\null
}

% 重新设置公式autoref
\def\equationautorefname~#1\null{%
	式~(#1)~\null
}
















%%%%%%%%%调整subsection与subsubsection格式
\titlespacing*{\subsubsection}{0pt}{0.5ex plus .2ex minus .2ex}{%
	0.5ex plus .2ex
}
\titlespacing*{\subsection}{0pt}{0.5ex plus .2ex minus .2ex}{%
	0.5ex plus .2ex
}
%%%%%%%%%貌似是调整参考文献格式?
\let\oldthebibliography \thebibliography
\let\endoldthebibliography \endthebibliography
%\renewenvironment{thebibliography}[1]{%
%	\begin{oldthebibliography}{#1}%
%		\setlength{\parskip}{0ex}%
%		\setlength{\itemsep}{0ex}%
%		\setlength{\itemindent}{4ex}
%		\setlength{\leftmargin}{-3pt}%
%	}% 
%	{
%	\end{oldthebibliography}%
%}
%%%%%%%%%调整数学公式格式,貌似是设置公式编号格式?
\renewcommand{\theequation}{\thesection -\arabic{equation}}
\makeatletter
%%%%%%%%%貌似是调整参考文献文本格式(如缩进等),非参考文献自身格式
\renewenvironment{thebibliography}[1]
{\section*{\refname}%
	\@mkboth{\MakeUppercase\refname}{\MakeUppercase\refname}%
	\list{\@biblabel{\@arabic\c@enumiv}}%
	{\settowidth\labelwidth{\@biblabel{#1}}%
		\setlength{\itemindent}{\dimexpr\labelwidth+\labelsep}
		\leftmargin\z@
		\@openbib@code
		\usecounter{enumiv}%
		\let\p@enumiv\@empty
		\renewcommand\theenumiv{\@arabic\c@enumiv}}%
	\sloppy
	\clubpenalty4000
	\@clubpenalty \clubpenalty
	\widowpenalty4000%
	\sfcode`\.\@m}
{\def\@noitemerr
	{\@latex@warning{Empty `thebibliography' environment}}%
	\endlist}

%%%%%貌似是数学公式从每个章节(section)重新开始编号
\@addtoreset{equation}{section}


\begin{document}
\fi
%%%%%---------------------------------------------------------------------



%\bibliography{./bibs/bibliography.bib}

%%%%%之前的.bib生成引用的样式与word要求不一致。能力有限,暂时无法修改
%%%%%样式文件不同于学校的要求。劳烦同学们手动引用文献

%%%%%手动指定在目录添加参考文献条目
\clearpage
\pagestyle{afterbody}
\phantomsection
\addcontentsline{toc}{section}{参考文献}

% 按照学校word模板中对参考文献的要求,列出以下几点给同学们参考


% 列出的参考文献必须在正文中有引用,并且需按正文中出现的次序进行排序。同一文献出现多次,只用同一标号

% 参考文献里的标点符号均为英文格式输入,每个标点符号与后面的内容之间要空一格。参考文献的各项条目使用逗号分割,最后要有句点。


% 参考文献应不少于10篇(外文文献至少2篇,外语专业应以外文文献为主)。

% 文献引用的格式大致为:

% 作者1, 作者2, 作者3, et. al, 题目, 期刊, 时间, 期数(卷数), 起-始页,网址.

% 其中不那么重要的或没有的部分可不写
% 其中:

% 英文作者,名缩写(老外是名在前,姓在后),如:Robert Jort缩写为:R. Jort,名字两个单词的,G. H. Golub。作者太多不适合全部列出的,写上 et. al,

% 英文题目除专有名词外,仅第一个单词首字母大写
% 题目中表示文献类型的符号:[M] [J] 等一律删掉,不允许出现。
% 期刊名应写全称,不知道的可以上网搜索。英文期刊中实词首字母的写。
% 一些英文常见期刊:

% 日期统一改为如下格式 2003.5.12


\begin{thebibliography}{99}
\bibitem{1} 严蔚敏, 吴伟民, 数据结构, 北京: 清华大学出版社, 1997.4.
\bibitem{2} 沈晴霓, 聂青, 苏京霞, 现代程序设计—C++与数据结构面向对象的方法与实现, 北京: 北京理工大学出版社, 2002.8.
\bibitem{3} T. Connolly, C. Begg, Database systems, 北京: 电子科技工业出版社, 2004.7.
\bibitem{4} R. Bate, S. Shrum, CMM Integration framework, CMU/SEI Spotlight, 1998, 4(3): 25-28.
\bibitem{5} J.P. Kuilboer, N. Ashrafi, Software process and product improvement, Physical Review A, 2000, 42(1): 27-34.
\bibitem{6} 张美金, 吴大伟, 基于ASP技术的远程教育系统体系结构的研究, http://172.50.0.88:86 /~cddbn/Y517807/pdf/index.htm, 2003-05-01.
\bibitem{7} 王伟国, 刘永萍, 王生年等, B/S模式网上考试系统分析与设计, 石河子大学学报(自然科学版), 2003, 6(2): 145-147.
\bibitem{8} …
\bibitem{9} …
\bibitem{10} …
\end{thebibliography}




%%%%%---------------------------------------------------------------------
\ifx\compileAllFiles\undefined
\end{document}
\fi